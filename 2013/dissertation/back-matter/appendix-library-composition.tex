\appendixchapter{Library Composition}
\label{appdx:library}

Library composition in fault tolerance is considered an especially difficult 
problem. To demonstrate the feasibility of our solution, this appendix includes a 
sample implementation of a hierarchy of libraries. This code demonstrates token 
libraries to scale and add two vectors. While the function of the libraries is 
not important, the initialization and recovery code within the libraries is the 
key contributed to be noted here. While this is certainly not the only possible 
implementation of a library and probably not even the most efficient, it is a 
good reference for developers as an example of how to construct their recovery 
mechanisms.

\appendixsection{Main application}

This is the main code of the application. When a failure occurs in one of the 
lower level libraries, they will return control to this library to perform high 
level recovery (including repairing the MPI communicators) and then call the 
repair functions for the low level libraries before returning to the partially 
completed function calls.

\subsubsection*{vector_math.c}

\lstinputlisting[language=C,basicstyle=\ttfamily,showspaces=false,breakatwhitespace=true,showstringspaces=false,showtabs=false,breaklines=true]{back-matter/onelevel.c}

\appendixsection{Library 1}

This section is the header and main code for the first library. This library 
performs the scaling operation. Note how the library tracks recovery status by 
using a status object which is actually managed by the calling code. This 
facilitates recover across instances in cases where a node may be migrated and 
the data recovered in a new location.

\subsubsection*{lib1.h}

\lstinputlisting[language=C,basicstyle=\ttfamily,showspaces=false,breakatwhitespace=true,showstringspaces=false,showtabs=false,breaklines=true]{back-matter/lib1.h}

\subsubsection*{lib1.c}

\lstinputlisting[language=C,basicstyle=\ttfamily,showspaces=false,breakatwhitespace=true,showstringspaces=false,showtabs=false,breaklines=true]{back-matter/lib1.c}

\appendixsection{Library 2}

This section is the header and main code for the second library. This library 
performs the addition operation. Again, note how the library tracks recovery status by 
using a status object which is actually managed by the calling code. This 
facilitates recover across instances in cases where a node may be migrated and 
the data recovered in a new location.

\subsubsection*{lib2.h}

\lstinputlisting[language=C,basicstyle=\ttfamily,showspaces=false,breakatwhitespace=true,showstringspaces=false,showtabs=false,breaklines=true]{back-matter/lib2.h}

\subsubsection*{lib2.c}

\lstinputlisting[language=C,basicstyle=\ttfamily,showspaces=false,breakatwhitespace=true,showstringspaces=false,showtabs=false,breaklines=true]{back-matter/lib2.c}
