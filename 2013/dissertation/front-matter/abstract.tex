\chapter*{Abstract}
\label{chap:abstract}

As machine sizes have increased and application runtimes have lengthened, 
research into fault tolerance has evolved alongside. Moving from result checking, 
to rollback recovery, and to algorithm based fault tolerance, the type of recovery 
being performed has changed, but the programming model in which it executes has remained 
virtually static since the publication of the original Message Passing Interface 
(MPI) Standard in 1992. Since that time, applications have used a message passing 
paradigm to communicate between processes, but they could not perform process 
recovery within an MPI implementation due to limitations of the MPI Standard. 
This dissertation describes a new protocol using the exiting MPI Standard called 
Checkpoint-on-Failure to perform limited fault tolerance within the current framework of MPI, 
and proposes a new platform titled User Level Failure Mitigation (ULFM) to build 
more complete and complex fault tolerance solutions with a true fault tolerant 
MPI implementation. We will demonstrate the overhead involved in using these 
fault tolerant solutions and give examples of applications and libraries which 
construct other fault tolerance mechanisms based on the constructs provided in 
ULFM.