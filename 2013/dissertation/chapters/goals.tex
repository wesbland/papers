\chapter{Design Goals}\label{chap:goals}

After evaluating the features, strengths and weaknesses of the previous research
in fault tolerance, four main goals for a successful fault tolerant communication library
emerge. While not all libraries will fulfill all of these goals, for an \mpi
library to be successful at supporting a wide variety of fault tolerant
paradigms, these are foundational principles which should be considered during 
the design.

\section{Flexibility}\label{sect:goals:flexibility}

A successful fault tolerant library must provide the flexibility to support
multiple consistency and recovery techniques. For example, a Monte-Carlo
master-worker application may not require complex recovery after a process
failure is detected by the master process. Rather, the failed worker process can
safely be ignored. If an \mpi implementation attempts to perform some method of
automatic recovery, it would not only introduce a high recovery cost to an
application which does not require it, but it would also require the application
to change its behavior in order to support the type of fault tolerance mandated
by the library. This is not the most flexible approach to fault tolerance and
therefore limits its usefulness as part of a communication standard.
Rather, the standard should provide the minimum level of recovery, only
enough to allow further communication, and then allow the application to choose
what direction subsequent recovery should take.

\section{Resilience}\label{sect:goals:resiliency}

Resilience refers not only to the ability of the MPI application to survive
failures, but also to recover into a consistent state from which the execution
can be resumed. This manifests most profoundly in the effort to ensure that an
MPI operation cannot stall indefinitely as a consequence of a failure. If an
operation never returns, the application can take no part in the recovery, and
fault tolerance is impossible. All operations which perform communication must
return descriptive error codes to inform the application of any unexpected
behavior which occurred while the library was executing. As long as some
processes in the application are informed of a failure, they can initiate
recovery actions. In addition to being deadlock-free, the library must also
provide mechanisms to alert other processes to failure when necessary. These
mechanisms could be automatic within the library or manual via an external construct.

\section{Performance}\label{sect:goals:performance}

The performance impact of any fault tolerance additions to an \mpi communication
library must be minimal when outside of the recovery path. Internal recovery
should be triggered only when necessary and normal failure monitoring actions
should take place out of the performance critical path. As mentioned in
Section~\ref{sect:goals:resiliency}, not only should the failure-free operations
introduce insignificant levels of overhead, but recovery operations should also be fast.
Many automatic fault tolerance techniques exhibit poor performance as they require 
universal participation in recovery after a failure. Rather than imposing such global
knowledge on the system, a minimal, local knowledge shows much more promise for
high performance. When alerted to a failure, if it is necessary to inform other
processes, appropriate constructs should be called by the application, not the
library, to ensure that only necessary levels of recovery are executed.

\section{Productivity}\label{sect:goals:productivity}

The last goal is harder to measure empirically, but is nonetheless critical in
the design of a fault tolerant \mpi library. An enormous number of legacy \mpi
codes already exist which do not support fault tolerance and would not benefit
from its support if it were to be implemented. To that end, any new fault
tolerance additions to the \mpi Standard must not require changes from such
legacy applications. This means that the behavior of existing \mpi operations
should not change without a severe need.

In addition, the fault tolerance constructs should be minimal both in terms of
quantity and complexity. By providing the minimal set of changes to \mpi,
the chances of the library being used increase and the time required to adopt
the library decreases.

When designing a minimal set of changes to supply fault tolerance, some 
convenience functions which might increase programmability will be left out. This 
does not prohibit such functions from existing. Instead, these functions may be 
provided as an external library built on the foundation of a minimal standard. 
These external libraries are not limited only to convenience functions. They can 
also introduce complex recovery mechanisms not found in a standardized document.
