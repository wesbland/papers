\section{Concluding Remarks}

In this paper, we presented an original scheme to deal with failures
using the current MPI standard, and still avoiding periodical
checkpointing. Periodical checkpointing is subject to a critical
parameter that is particularly hard to assess: the ideal period of
checkpoint. A period too short will loose time and resources, doing
unnecessary Input/Output. A period too long will also loose time and
resources, by increasing the amount of the execution that must be
re-done, if a failure hits the system a long time after the last
successful checkpoint.

On-Demand Checkpointing takes checkpoint images at optimal times by
design: only after a failure has been detected. It relies on \abft
techniques to complement this checkpoint with redundant information
and enable recovering the whole application data at restart
time. This simple scheme is easily integrable with high quality MPI
implementations, as long as they let the application survive (even
with MPI non-functionning) when a failure is detected and reported. We
proposed performance models to evaluate and compare the benefits and
limitations of On-Demand Checkpointing and periodical
Checkpoint. These models highlight another feature of On-Demande
Checkpointing: since the checkpoint images of the processes that were
not subject to failure is needed to recover the state, local disks can
be used, providing a much higher bandwidth than the remote storage
needed by periodical Checkpointing. Last, but not least of the
advantages of On-Demand Checkpointing, the restart does not trigger a
rollback recovery, since the recovery procedure is able to reconstruct
the data at the moment of the crash.  The performance evaluation,
based on the \abft version of the QR factorization show that the
disk Input/Output required by the method after a fault occured does
not prevent to obtain excellent performance, and that a high quality
MPI implementation can be obtained without decreasing its efficiency.
