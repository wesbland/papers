\section{Related Work}\label{sect:related}

% Why coordinated checkpointing is no longer scalable

The current industry standard for failure handling is rollback recovery with
periodic checkpoints to disk, and many libraries have been implemented to
support this behavior~\cite{Duell:tr, Litzkow:1997wd, Plank:1994wz,
Zhong:2001tq}. This has been an effective recovery model for many years and
continues to be an area of research to prolong its
usefulness~\cite{BautistaGomez:2011hg, BuntinasFGCS2008, Elnozahy:2002p3769},
despite the increase in fault frequency and the hierarchization of the
underlying hardware architecture.  However as machines continue to scale,
concerns have been raised about the scalability of rollback
recovery~\cite{Cappello:2009hs}.  The time spent performing recovery operations
is expected to exceed the MTBF in the next generation of supercomputers, causing
any large-scale applications to enter a cycle of recovery where no useful
computation occurs.

% Resilience workshop report

In February 2012, the Department of Defense and Department of Energy conducted
research~\cite{Daly:2012vg} to outline the necessity for resilience at extreme
scale, specifically for exascale computing. They affirmed the likelihood that
exascale systems will have a shorter MTBF than existing systems. They also
recommended that ``a `light-weight' approach, i.e., effective and easy to
implement, is preferable to a `full-featured' alternative.'' By attempting fewer
features, each library can focus on creating a scalable and efficient
implementation while promoting simplicity and reducing unnecessary features. The
agencies also discussed possible redundancy approaches (discussed
in~\cite{BosilcaINRIARep7950, Ferreira:2011fb}), describing them as ``not
practical because they require 2-3x more energy''. The price of redundancy must
not only include the obvious loss of computing time from executing duplicate
codes across multiple processors, but also the up-front costs of purchasing
extra hardware on which to execute the redundancy.

% Describe ABFT

Out of these problems have arisen a new class of algorithms that allow different
methods of resilience. These algorithms support what is called Algorithm Based
Fault Tolerance (ABFT). They have been designed to support a recovery model that
does not require all processes to participate in the recovery simultaneously or,
perhaps, ever at all. Huang and Abraham first explored these
algorithms~\cite{Huang:1984kt} for soft errors in systolic arrays, but research
has expanded to create ABFT techniques for many algorithms~\cite{Du:2012je, Plank:1995hv}.

% RTS Proposal

To support these new algorithms, applications require support from their
libraries. One of the most popular communication libraries used by large scale
applications is the Message Passing Interface (MPI), the contents of which are
determined by the MPI Forum. The MPI Forum has previously considered proposals
to add fault tolerance capabilities to the MPI Standard. In 2011, a proposal was
brought forth~\cite{Hursey2011RTS} titled ``Run-Through Stabilization'' which
discussed a wide range of fault tolerance capabilities to provide for users by
defining the behavior of MPI following a process failure as well as introducing
new constructs to provide strong consensus between processes. The MPI Forum
decided to reject the proposal, but the work led to more research in the field
of resilience in MPI, including the work being proposed in
Section~\ref{sect:ulfm}.

% FT-MPI

Outside of the MPI Forum, work has been done to provide fault tolerance to MPI
applications. FT-MPI~\cite{FaggFTMPI} was an MPI-1 compliant implementation of
the MPI Standard which added new capabilities for fault tolerance. It included
automatic recovery models for failures including: shrinking MPI communicators to
automatically remove failed processes, leaving holes in the communicators while
allowing communication to continue, and destroying and rebuilding the
communicators to retain communication patterns and groups. FT-MPI was never
adopted into the MPI Standard but continued as a research project for many
years. 
