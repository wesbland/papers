%\section{Related Works}\label{section:related}

%\paragraph{{\bf Related Work}}

Some efforts have been undertaken to enable \abft support in MPI. 
FT-MPI~\cite{fagg2000ft} was an MPI-1 implementation which proposed to
change the MPI semantic to enable repairing communicators, thus
re-enabling communications for applications damaged by failures. This
approach has proven successful and applications have been
implemented using FT-MPI. However, these modifications were not adopted by
the MPI standardization body, and the resulting lack of portability
undermined user adoption for this fault tolerant solution.

%  One of
% the solutions implemented in FT-MPI was to introduce a new MPI\_Errhandler
% called MPI\_ERRORS\_BLANK. This MPI\_Errhandler replaced the position of the
% failed processes inside a communicator with MPI\_PROC\_NULL. By using this
% semantic, the remaining MPI calls could function normally as communication with
% MPI\_PROC\_NULL always succeeds. When FT-MPI encountered a fault, it destroyed
% all MPI Communicators and required that the application recreate them to account
% for the failed processes. While this was a useful step in allowing the most
% level of flexibility from the application's perspective, it made recovery very
% complex and added a large overhead.

% FT-MPI is another form of fault tolerance that can be successful for some, but in
% applications that require that all processes be running to reach successful
% completion, having a hole in the communicator is not a valid solution. The
% failed processes need to somehow be recovered.

%\TBD{Anything needed from the QR side of things?}

In %the article by W. Gropp and E. Lusk in 2004
\cite{Gropp:2004:FTM:1080704.1080714}, the authors discuss alternative
or slightly modified interpretations of the MPI standard that enable
some forms of fault tolerance. 
%They consider different approaches,
%including periodical checkpointing, and using inter-communicators to
%contain error propagation between separate worlds of an MPI application.
One core idea is that process failures happening in another MPI world, connected only through an inter-communicator, should not prevent the continuation of normal operations. The complexity of this approach, for both the implementation
and users, has prevented these ideas from having a practical impact. In the \cof
approach, the only requirement from the MPI implementation is that it does not
forcibly kill the living processes without returning control. No
stronger support from the MPI stack is required, and the state of the
library is left undefined. This simplicity
has enabled us to actually implement our proposition, and then experimentally support and
evaluate a real \abft application. Similarly, little effort would be required to extend 
MPICH-2 to support \cof (see Section 7 of the Readme\footnote{http://www.mcs.anl.gov/research/projects/mpich2/documentation/files/mpich2-1.4.1-README.txt}).

