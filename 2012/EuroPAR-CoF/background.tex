\section{Background \& Related Work}
%\section{Background}
\label{sect:background}

% Discuss some MPI-3 related proposals and issues

%Among the issues
%raised during the readings of the proposals, were the fact that these
%approaches will still incur a significant overhead on failure free
%operations, by requiring periodic \emph{consensus}

%\paragraph{{\bf Background}}

Message passing is the dominant form of communication used in parallel
applications, and MPI is the most popular library used to implement it.
%However,as fault tolerance becomes a growing concern for application
%developers, users have encountered some challenges with the current MPI
%Standard that limit their options of fault tolerance methods. 
In this context, the primary form of fault tolerance today is rollback recovery with periodical checkpoints to disk. While this method is effective in allowing
applications to recover from failures using a previously saved state, it causes
serious scalability concerns~\cite{ExaScaleResilience09}. Moreover, periodic
checkpointing requires precise heuristics for fault frequency to minimize the
number of superfluous, expensive protective
actions~\cite{Young:1974,Gelenbe:1979,Plank01,Daly:2006,PreventiveCheckpointing11}.
In contrast, the work presented here focuses on enabling {\it forward
  recovery}. Checkpoint actions are taken only {\it after} a failure is
detected; hence the checkpoint interval is optimal by definition, as there will
be one checkpoint interval per effective fault.

Forward recovery leverages algorithms properties to complete operations
despite failures. In Naturally Fault Tolerant applications, the algorithm
can compute the solution while totally ignoring the contributions of failed processes. In~\abft applications, a recovery phase is necessary, but failure damaged data can be reconstructed only by applying mathematical operations on the remaining dataset~\cite{huang1984algorithm}. A recoverable dataset is usually created by initially computing redundant data, dispatched so as to avoid unrecoverable loss of information from failures. At each iteration, the algorithm applies the necessary mathematical transformations to update the redundant data (at the expense of more communication and computation). Despite great scalability and low overhead~\cite{luk1988analysis,pengduppopp12}, the
adoption of such algorithms has been hindered by the requirement that
the support environment must continue to consistently deliver
communications, even after being crippled by failures.

The current MPI Standard (MPI-2.2,~\cite{MPI22}) does not provide
significant help to deal with the required type of behavior. Section~2.8
states in the first paragraph: ``\emph{MPI does not provide mechanisms
  for dealing with failures in the communication system. [\ldots]
  Whenever possible, such failures will be reflected as errors in the
  relevant communication call. Similarly, MPI itself provides no
  mechanisms for handling processor failures.}'' Failures, be they due
to a broken link or a dead process, are considered  resource
errors. Later, in the same section: ``\emph{This document does not
  specify the state of a computation after an erroneous MPI call has
  occurred. The desired behavior is that a relevant error code be
  returned, and the effect of the error be localized to the greatest
  possible extent.}'' So, for the current standard, process or
communication failures are to be handled as errors, and the behavior
of the MPI application, after an error has been returned, is left
unspecified by the standard. However, the standard does not prevent
implementations from going beyond its requirements, and on the contrary,
encourages high-quality implementations \emph{to return} errors, once a
failure is detected. Unfortunately, most of the implementations of the 
MPI Standard have taken the path of considering process failures as 
unrecoverable errors, and the processes of the application are most 
often killed by the runtime system when a failure hits any of them, 
leaving no opportunity for the user to mitigate the impact of 
failures. 

%The MPI forum is currently examining options for the future direction
%of MPI for MPI-3. One of the workgroups is dedicated to propose a
%standard form of MPI-supported fault tolerance. The proposal outlines
%a method of run-through stabilization which allows the application to
%acknowledge and repair communications, both collectively and between
%specific ranks in a point-to-point way~\cite{Hursey11MPI3FT}. The
%emphasis of the proposal is a set of "validation" functions which the
%application is required to call to repair and re-enable communication within
%an MPI communicator containing a failed process. To repair point to
%point wildcard receives, the application needs to collectively call the function
%MPI\_COMM\_REENABLE\_ANY\_SOURCE. To repair collective communication
%within a communicator, the application needs to call the function
%MPI\_COMM\_VALIDATE.  These functions give the MPI implementation an
%opportunity to acknowledge failures and discover or ensure that other
%MPI processes also acknowledge the same failures. It also gives the
%MPI library a chance to repair communication channels between
%remaining processes, optimizing communication topologies if possible
%and necessary.
%
%While this method of fault tolerance is sufficient for \abft, it is
%not without its drawbacks. The calls necessary to recover from
%collectives incur a non-trivial overhead even during the fault free
%case. MPI\_COMM\_VALIDATE requires a distributed consensus algorithm
%which is currently best implemented at log
%scale~\cite{Hursey11LogConsensus}. While this level of overhead is
%better than the current state of the art of periodic checkpointing, it
%still presents a significant cost that not all applications want or
%need to pay to check the validity of the communicators. Most
%importantly, this proposal does not yet include process recovery,
%which is left to a future proposal to the MPI forum.

% Discuss issues in general with FT-MPI like approaches, besides the 
% sheer problem of standard adoption

% Explain why it is not believed that ABFT can perform without REPLACE 
% or BLANK, or leave it for next section ?

\section{Related Work}
\label{sect:related}

Efforts toward fault tolerance in MPI have previously been attempted.  Automatic
fault tolerance~\cite{Bouteiller10Redesign,MPICHVblocking} is a compelling
approach for users, as failures are completely masked and handled internally by
the MPI library, which requires no new interfaces to MPI or application code
changes. Unfortunately, many recent studies point out that automatic approaches,
either based on checkpoints or replication, will exhibit poor efficiency on
Exaflop platforms~\cite{BOSILCA-2012-696154,lawn265}.

Application Based Fault Tolerance
(ABFT)~\cite{fthpl2011,pengduppopp12,huang1984algorithm} is another approach
that promises better scalability, at the cost of significant algorithm and 
application code changes. Despite some limited 
successes~\cite{europar12/onfailureckpt,Gropp:2004:FTM:1080704.1080714},
MPI interfaces need to be extended to effectively support ABFT. The most notable past 
effort is FT-MPI~\cite{fagg2000ft}, which provided several recovery modes for the user 
but was never standardized and therefore was not portable enough for most users.
