\section{Motivation} \label{sect:intro}

Fault tolerance is an increasingly necessary consideration in High Performance Computing (HPC). As machine sizes increase past hundreds of thousands of computing cores\footnote{http://www.top500.org} into the millions of computing resources, the likelihood of failures also increases. Observed failures rates are reaching between 1.8 and 3.6 failures per day on a system of only 635 nodes. This research confirms what has become an accepted reality of HPC going forward. Failures will occur at an increasing rate and for large scale applications to be useful, the failures will need to be handled in software while allowing the applications to continue running relatively uninterrupted.

This realization has lead to much research to attempt to solve the problems presented by the necessity for fault tolerance. The first and most well understood form of fault tolerance is rollback-recovery using periodic checkpointing. This form of fault tolerance has been widely adopted and works well for small scale machines where failure rates are expected to be relatively low. However, at larger scales, even with more reliable hardware, the  time spent performing checkpointing operations is expected to exceed the amount of time spent performing useful computation. To resolve this problem, we turn to Application Based Fault Tolerance (ABFT).

ABFT changes the way applications recover from failure. Rather than loading a previous checkpoint from disk and restarting an entire application, the algorithm itself recovers from the loss of a process and continues without the need to perform costly, large-scale checkpointing operations. The exact method used to recover from failures changes from application to application, but all applications have some requirements in common. The programming model and environment, together with the supporting runtime, need to provide basic functionality in order to allow applications to build a comprehensive fault management solution.

This is a challenging requirement. Most applications continue to use message passing to perform communication between processes, and in the message passing paradigm, collective communication is a popular and necessary way for groups of processes to communicate efficiently. However, collective communication also creates problems when trying to maintain the functionality of the communication library following a process failure due to the complex communication patterns and topologies that must be repaired. 

In addition to maintaining a functional communication library with scalable fault tolerance mechanisms, a fault tolerance solution must also provide extensibility. Because ABFT takes different forms for different applications, the fault tolerance provided by the communication library should also be able to adapt. For example, while one type of application may work best with a transactional model of fault tolerance where sections of the application are re-executed when recovery is necessary, a master-slave type of application may choose to simply spawn a replacement for any process which fails and continue on without needing further recovery.

To this end, we have modified a runtime system to support two new forms of fault tolerance within the message passing paradigm which can provide a suite of tools necessary for developers to include resilience in their applications. Due to size restrictions, details on the work implementing a resilient runtime can be found in~\cite{Bland:CCGrid12}.