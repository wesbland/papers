\section{Conclusion}
\label{sect:conclusion}

\vspace{-0.1cm}Many responsible voices agree that sharp increases in the volatility of future,
extreme scale computing platforms are likely to imperil our ability to use them
for advanced applications that deliver meaningful scientific results and
maximize research productivity. Moreover, it is clear that any techniques
developed to address this volatility must be supported in the programming
and execution model. Since MPI is currently, and will likely continue to be --
in the medium-term -- both the de-facto programming model for distributed
applications and the default execution model for large scale platforms running
at the bleeding edge, MPI is the place in the software infrastructure where semantic
and run-time support for application faults needs to be provided.

The \ulfm proposal is a careful but important step forward toward accomplishing
this goal.  It not only delivers support for a number of new and innovative
resilience techniques, it provides this support through a simple,
straightforward and familiar API that requires minimal modifications of the
underlying MPI implementation. Moreover, it is backward compatible with previous
versions of the MPI standard, so that non fault-tolerant applications (legacy or
otherwise) are supported without any changes to the code. Perhaps most
significantly, applications can use \ulfm-enabled MPI without experiencing any
degradation in their performance, as we demonstrate in this paper.

Several applications, ranging from Master-Worker to tightly coupled, are
currently being refactored to take advantage of the semantics in this
proposal. Beyond applications, the expressivity of this proposal is 
investigated in the context of providing extended fault tolerance 
models as convenience, portable libraries.
