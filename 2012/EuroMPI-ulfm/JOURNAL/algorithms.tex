\section{Implementation Issues}
\label{sect:algorithms}

In this section, we detail the challenges and advantages of the aforementioned
MPI constructs. They unfold along three main axes, the amount of supplementary
state and memory to be kept within the MPI library, the additional operations to
be executed on the critical path of communication routines, and the algorithmic
cost of failure recovery routines. We discuss, in general, options available to
implementors, and highlight issues with insight from a prototype implementation
in Open~MPI~\cite{gabriel04:_open_mpi}.

\subsection{Impact on communication routines}

\paragraph*{Memory:} Because a communicator cannot be repaired,
tracking the state of failed processes imposes a minimal memory overhead.  From
a practical perspective each node needs a global list of detected failures,
shared by all communicators; its size grows linearly with the number of
failures, and it is empty as long as no failures occur.  Within each
communicator, the supplementary state is limited to two values: whether the
communicator is revoked or not, and an index in the global list of failures
denoting the last acknowledged failure (with
\mpifunc{MPI\_COMM\_FAILURE\_ACK}). For efficiency reasons, an implementation
may decide to cache the fact that some failures have happened in the
communicator so that collective operations and \mpifunc{MPI\_ANY\_SOURCE}
receptions can bail out quickly. Overall, the supplementary memory consumption
from fault tolerant constructs is small, independent of the total number of
nodes, and unlikely to affect the cache and TLB hit rates.

\paragraph*{Conditionals:} Another concern is the number of supplementary
conditions on the latency critical path. Indeed, most completion operations
require a supplementary conditional statement to handle the case where the
underlying communication context has been revoked. However, the prediction
branching logic of the processor can be hinted to favor the failure free
outcome, resulting in a single load of a cached value and a single, mostly
well-predicted, branching instruction, unlikely to affect the instruction
pipeline. It is notable that non-blocking operations raise errors related to
process failure only during the completion step, and thus do not need to check
for revocation before the latency critical section.

\paragraph*{Matching logic:} \mpifunc{MPI\_COMM\_REVOKE} does not have a
matching call on other processes on which it has an effect. As such, it might
add detrimental complexity to the matching logic. However, any MPI
implementation needs to handle unexpected messages. The order of revocation
message delivery is loose enough that the handling of revocation notices can be
integrated within the existing unexpected message matching logic. In our
implementation in Open~MPI, we leverage the active message low level transport
layer to introduce revocation as a new active message tag, without a single
change to the matching logic.

\paragraph*{Collective operations:} A typical MPI implementation supports a
large number of collective algorithms, which are dynamically selected depending
on criteria such as communicator or message size and hardware topology. The
loose requirements of the proposal concerning error reporting of process
failures in collective operations limits the impact it has on collective
operations. Typically, the collective communication algorithms and selection
logic are left unchanged. The only new requirement is that failures happening at
any rank of the communicator cause all processes to exit the collective
(successfully for some, with an error for others). Due to the underlying
loosely-connected topologies used by some algorithms, a point-to-point based
implementation of a collective communication is unlikely to detect all process
failures.  Fortunately, a practical implementation exists that does not require
modifying any of the collective operations: when a rank raises an error because
of a process failure, it can revoke an internal, temporary communication
context associated with the collective operation. As the revocation notice
propagates on the internal communicator, it interrupts the point-to-point
operations of the collective. An error code is returned to the high level MPI
wrapper, which in turn raises the appropriate error on the user's communicator.

\subsection{Recovery routines}

Some of the recovery routines described in Section~\ref{sect:newmpi} are
unique in their ability to deliver a valid result despite the occurrence
of failures. This specification of correct behavior across failures calls
for resilient, more complex algorithms. In most cases, these functions
are intended to be called sparingly by users, only after actual failures have
happened, as a means of recovering a consistent state across all
processes. The remainder of this section describes the algorithms that
can be used to deliver this specification and their cost.

\paragraph*{Agreement:}
\label{subsect:agreement}

The agreement can be conceptualized as a failure-resilient reduction on
a boolean value. Many agreement algorithms have been proposed in the 
literature; the log-scaling two-phase consensus algorithm used by the \ulfm
prototype is one of many possible implementations of
\mpifunc{MPI\_COMM\_AGREE} operation based upon prior work in the field.
Specifically, this algorithm is a variation of the multi-level two-phase
commit algorithms~\cite{Mohan_1985}. 
The algorithm first performs a reduction of the input values to an
elected coordinator in the communicator. The coordinator then makes a
decision on the output value and broadcasts that value back to all of
the alive processes in the communicator. The complexity of the agreement
algorithm appears when adapting to an emerging process failure of the
coordinator and/or participants. A more extensive discussion of the
algorithmic complexity has been published by Hursey,
et.al.~\cite{Hursey11LogConsensus}. The algorithmic complexity of this
implementation is $O(log(n))$ for the failure free case, matching that
of an \mpifunc{MPI\_ALLREDUCE} operation over the alive processes in the
communicator.

\paragraph*{Revoke:}
\label{subsect:revoke}

Although the revoke operation is not collective, the revocation notification
needs to be propagated to all alive processes in the specified communicator,
even when new failures happen during the revoke propagation.  These requirements
are not without recalling those from the \emph{reliable
  broadcast}~\cite{Hadzilacos:1993:FBR:302430.302435}.  Among the four defining
qualities of a reliable broadcast (\emph{Termination, Validity, Integrity,
  Agreement}), the termination and integrity criteria can be relaxed in the
context of the revoke algorithm. If a failure during the Revoke
algorithm kills the initiator as well as all the already notified processes, the Revoke
notification is indeed lost, but the observed behavior, from the view of the
application, is indiscernible from a failure at the initiator before the
propagation started. As the algorithm still ensures agreement, there are no
opportunities for inconsistent views.

%In the \ulfm implementation, we used a naive flooding algorithm for simplicity.
%The initiator marks the communicator as revoked and sends a Revoke message to
%every processes in the groups (local and remote) of the communicator.  Upon
%reception of a revoke message, if the communicator is not already revoked, it is
%revoked and the process acts as a new initiator.
%Better algorithms exist, but even this naive approach provides reasonable
%performance (see Section~\ref{sect:performance}) considering it is called only
%in response to an actual failure.
%

In the \ulfm implementation, we use a Binomial Graph
(BMG)~\cite{Angskun07binomialgraph}. The initiator marks the
communicator as revoked, and sends a Revoke message to $log(n)$ other
processes (in a communicator of size $n$). Upon the reception of a
revoke message, if the communicator is not already revoked, it is then
revoked and the process acts as a new initiator. This topology has the
advantage of featuring a lower message count ($O(n log(n))$ messages are
exchanged, instead of $O(n^2)$). Because the revoke algorithm is
asynchronous, the performance benefit, as observed by applications, is
low. However, complete flooding was suffering from a practical problem
at scale: it required opening a large number of connections, resulting
in crashing the Open IB driver. The BMG topology is protected against
that defect, because in this topology, the degree per node is
$O(log(n))$; in the experiments, no crashes were observed. The converse
drawback is that if $log(n)$ simultaneous failures strike in an exact
pattern, and the failure had been detected by only one process, the
revoke algorithm could fail to deliver the notification to some alive
processes. The probability of such an outcome is very low (it is a
generalization of the birthday problem for $log(n)$ simultaneous
birthdays), and we are confident that it will never be observed in
practice. 


\paragraph*{Shrink:} The Shrink operation is, algorithmically, an agreement on
which the consensus is done on the group of failed processes.
Hence, the two operations have the same algorithmic complexity. Indeed, in the
prototype implementation, \mpifunc{MPI\_COMM\_AGREE} and
\mpifunc{MPI\_COMM\_SHRINK} share the same internal implementation of the
agreement.
