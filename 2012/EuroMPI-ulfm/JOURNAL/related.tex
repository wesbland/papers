\section{Related Work}
\label{sect:related}

Efforts toward fault tolerance in MPI have previously been attempted.  Automatic
fault tolerance~\cite{Bouteiller10Redesign,MPICHVblocking} is a compelling
approach for users, as failures are completely masked and handled internally by
the MPI library, which requires no new interfaces to MPI or application code
changes. Unfortunately, many recent studies point out that automatic approaches,
either based on checkpoints or replication, will exhibit poor efficiency on
Exaflop platforms~\cite{BOSILCA-2012-696154,lawn265}.

Application Based Fault Tolerance
(ABFT)~\cite{fthpl2011,pengduppopp12,huang1984algorithm} is another approach
that promises better scalability, at the cost of significant algorithm and application code changes. 
Despite some limited successes~\cite{europar12/onfailureckpt,Gropp:2004:FTM:1080704.1080714},
MPI interfaces need to be extended to effectively support ABFT. 
The most notable past effort is
FT-MPI~\cite{fagg2000ft}. Several recovery modes were available to the user. In
the \emph{Blank} mode, failed processes were replaced by
\mpifunc{MPI\_PROC\_NULL}; messages to and from them were silently ignored and
collective algorithms had to be significantly modified.
In the \emph{Replace} mode, faulty processes were replaced with new
processes. In all cases, only \mpifunc{MPI\_COMM\_WORLD} would be repaired and
the application was in charge of rebuilding any other communicators, leading to
difficult library composition. No standardization effort was pursued, and
it was mostly used as a playground for understanding the
fundamental concepts.

A more recent effort to introduce failure handling mechanisms was the
Run-Through Stabilization proposal~\cite{Hursey11MPI3FT}.
This proposal introduced many new constructs for MPI including the ability to
``validate'' communicators as a way of marking failure as recognized and
allowing the application to continue using the communicator. It included other
new ideas such as Failure Handlers for uniform failure notification. Because
of the implementation complexity imposed by resuming
operations on failed communicators, this proposal was eventually unsuccessful in
its introduction to the MPI Standard.

