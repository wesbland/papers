\section{Concluding Remarks \& Future Work} \label{sect:conclusion}

This work has outlined some of the improvements being performed within \ompi to
create a fault tolerant environment in order to allow applications to continue
their execution in the face of process failures. Our work requires modifications
to the runtime environment to stabilize message routing and provide reliable
fault detection and notification.

%The work will continue on as the MPI layer of \ompi needs to be modified to
%include support for the updated runtime layer, as well as to provide new
%capabilities not already described in the current MPI standard. This work
%includes creating fault tolerant collective communications patterns and a
%notification system to alert applications to the status of other processes
%within the application.
%
%As work progresses within \ompi, the resilient runtime layer will also improve.
%As an example, the current implementation of the runtime layer cannot withstand
%a failure of the manager, the HNP. This is not a major problem as this is the
%only node that is required to remain alive and while the probability of a
%failure occurring somewhere within the system might be high, the probability of
%a failure occurring specifically on the HNP is relatively low. This requirement
%could change as one of the potential projects within \ompi is to create a
%distributed HNP, removing the single point of failure.

These improvements provide the tools necessary for application developers to
implement algorithms that can run reliably at a larger scale than is currently 
possible due to the inability to handle the inherent failures which occur at
such extreme scales. Developers have new and more diverse options to allow them
to choose the appropriate resilience method for their application.

In the near future a system for process recovery will be included to
allow applications not only to stabilize themselves after a process failure, but
to replace the failed process with a new one. Such a process can use other forms of fault
tolerance (checkpointing, message logging, etc.) to recover lost data and
continue with minimal interruption to the living processes.

