\section{New MPI Constructs}
\label{sect:newmpi}

\paragraph{\mpifunc{MPI\_COMM\_FAILURE\_ACK} \& \mpifunc{MPI\_COMM\_FAILURE\_GET\_ACKED}:}\label{sect.ack}

These two calls allow the application to determine which processes within a
communicator have failed. The acknowledgement function serves to mark a point in
time used as a reference for the second function which returns the group of
processes which were locally know to have failed. After acknowledging failures,
the application can resume \mpifunc{MPI\_ANY\_SOURCE} point-to-point operations
between non-failed processes, but operations involving failed processes will
continue to raise errors.

\paragraph{\mpifunc{MPI\_COMM\_REVOKE}:}

For scalability purposes, failure detection is local to a process's neighbors as
defined by the application's communication pattern. This non-global error
reporting may result in some processes continuing their normal, failure-free
execution path, while others have diverged to the recovery execution path. As an
example, if a process, unaware of the failure, posts a reception from another
process that has switched to the recovery path, the matching send will never be
posted and the receive operation will deadlock.  The revoke operation
provides a mechanism for the application to resolve such situations before
entering the recovery path. A revoked communicator becomes improper for further
communication, and all future or pending communications on this communicator
will be interrupted and completed with the new error code
\mpifunc{MPI\_ERR\_REVOKED}.

\paragraph{\mpifunc{MPI\_COMM\_SHRINK}:}

The shrink operation allows the application to create a new communicator by
eliminating all failed processes from a revoked communicator. The operation is
collective and performs a consensus algorithm to ensure that all participating
processes complete the operation with equivalent groups in the new communicator.
This function cannot return an error due to process failure. Instead, such
errors are absorbed as part of the consensus algorithms and will be excluded
from the resulting communicator.

\paragraph{\mpifunc{MPI\_COMM\_AGREE}:}

This operation provides an agreement algorithm which can be used to determine a
consistent state between processes when such strong consistency is
necessary. The function is collective and forms an agreement over a boolean
value, even when failures have happened or the communicator has been
revoked. The agreement can be used to resolve a number of consistency issues
after a failure, such as uniform completion of an algorithmic phase or
collective operation, or as a key building block for strongly consistent failure
handling approaches (such as transactions).
