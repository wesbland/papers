\section{Motivation}
\label{sect:motivation}

A major concept followed in the design of any type of parallel programming paradigm, and
this in order to provide a meaningful and understandable programming approach,
is the avoidance of any potential deadlocks. In addition to this critical
requirement the mechanisms involved in the fault management were build with the
goal of flexibility, simplicity and performance.

Clearly the most important point, no MPI call (point-to-point or collective) can
block indefinitely after a failure, but must either succeed or raise an MPI
error. Fault tolerance at the application level is impossible if the application
cannot regain full control of the execution after a process failure. The MPI
library must guarantee that it will automatically stabilize itself following a
process failure, and provide the tools necessary for the application to resolve
its own deadlock scenarios on an application specific basis.
% The \ulfm proposal implements this idea with the \mpifunc{MPI\_COMM\_REVOKE}
% construct, detailed in Section~\ref{sect:newmpi}.

Second, the API should allow varied fault tolerant models to be built. MPI has
been conceived with the goal of portability and extendability, and have been
constructed to support libraries leveraging existing MPI constructs to create
more abstractions or tighter integration with libraries. Maintaining this design
strength was paramount for \ulfm, and it provides this capability so other
levels of consistency can be supported as needed by higher-level
concepts. Transactional fault tolerance, strongly uniform collective operations,
and other FT techniques can all therefore be built upon the proposed set of
constructs.

An API should be easy to understand and use in common scenarios, as complex
tools have a steep learning courve and a slow adoption by the targeted
communities. To this end, the number of newly proposed constructs have been
reduced to five (along with nonblocking variants). These five functions provide
the minimal set of tools to implement fault tolerant applications and libraries.

Two major pitfalls must be avoided when implementing these concepts: jitter
prone, permanent monitoring of the health of peers a process is not actively
communicating with, and expensive consensus required for returning consistent
errors at all ranks.  The operative principle is that fault-related errors are
local knowledge, and are not indicative of the return status on remote
processes. Errors are raised at a particular rank, when based on local knowledge
it is known that a particular operation cannot complete because a participating
peer has failed.
