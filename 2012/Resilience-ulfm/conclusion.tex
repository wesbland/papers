\section{Conclusion}
\label{sect:conclusion}

Many responsible voices agree that sharp increases in the volatility of future,
extreme scale computing platforms are likely to imperil our ability to use them
for advanced applications that deliver meaningful scientific results and
maximize research productivity. Since MPI is currently, and will likely continue
to be -- in the medium-term -- both the de-facto programming model for
distributed applications and the default execution model for large scale
platforms running at the bleeding edge, it is the place in the software
infrastructure where semantic and run-time support for application faults needs
to be provided.

The \ulfm proposal is a careful but important step forward toward accomplishing
this goal delivering support for a number of new and innovative resilience
techniques through simple, familiar API calls, but it is backward compatible
with previous versions of the MPI standard, so that non fault-tolerant
applications (legacy or otherwise) are supported without any changes to the
code. Perhaps most significantly, applications can use \ulfm-enabled MPI without
experiencing any degradation in their performance, as we demonstrate in this
paper. Some of these applications along with other portable libraries are
currently being refactored to take advantage of \ulfm semantics.

The author would like to acknowledge his co-authors in the full
paper~\cite{Bland:2012tp}: Aurelien Bouteiller, Thomas Herault, Joshua Hursey,
George Bosilca, and Jack J. Dongarra.
