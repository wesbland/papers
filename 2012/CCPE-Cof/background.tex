\section{Background \& Related Work}
%\section{Background}
\label{sect:background}

% Discuss some MPI-3 related proposals and issues

%Among the issues
%raised during the readings of the proposals, were the fact that these
%approaches will still incur a significant overhead on failure free
%operations, by requiring periodic \emph{consensus}

%\paragraph{{\bf Background}}

Message passing is the dominant form of communication used in parallel
applications, and the MPI standard specification, with its widely 
available implementations, forms the backbone of the HPC software 
infrastructure.
%However,as fault tolerance becomes a growing concern for application
%developers, users have encountered some challenges with the current MPI
%Standard that limit their options of fault tolerance methods. 
In this context, the primary form of fault tolerance today is rollback recovery
with periodical checkpoints to disk. While this method is effective in allowing
applications to recover from failures using a previously saved state, it causes
serious scalability concerns~\cite{ExaScaleResilience09}. Moreover, periodic
checkpointing requires precise heuristics for fault frequency to minimize the
number of superfluous, expensive protective
actions~\cite{Young:1974,Gelenbe:1979,Plank01,Daly:2006,PreventiveCheckpointing11}.
In contrast, the work presented here focuses on enabling {\it forward
  recovery}. Checkpoint actions are taken only {\it after} a failure is
detected; hence the checkpoint interval is optimal by definition, as there will
be one checkpoint interval per effective fault.

Forward recovery leverages algorithms' properties to complete operations despite
failures. In naturally fault tolerant applications, the algorithm can compute
the solution while totally ignoring the contributions of failed
processes. In~\abft applications, a recovery phase is necessary, but failure
damaged data can be reconstructed only by applying mathematical operations on
the remaining dataset~\cite{huang1984algorithm}. A recoverable dataset is
usually created by initially computing redundant data, dispatched so as to avoid
unrecoverable loss of information from failures. At each iteration, the
algorithm applies the necessary mathematical transformations to update the
redundant data (at the expense of more communication and computation). Despite
great scalability and low overhead~\cite{luk1988analysis,pengduppopp12}, the
adoption of such algorithms has been hindered by the requirement that the
support environment must continue to consistently deliver communications, even
after being crippled by failures.

The current MPI Standard (MPI-3.0,~\cite{MPI30}) does not provide
significant help to deal with the required type of behavior. Section~2.8
states in the first paragraph: ``\emph{MPI does not provide mechanisms
  for dealing with failures in the communication system. [\ldots]
  Whenever possible, such failures will be reflected as errors in the
  relevant communication call. Similarly, MPI itself provides no
  mechanisms for handling processor failures.}'' Failures, be they due
to a broken link or a dead process, are considered  resource
errors. Later, in the same section: ``\emph{This document does not
  specify the state of a computation after an erroneous MPI call has
  occurred. The desired behavior is that a relevant error code be
  returned, and the effect of the error be localized to the greatest
  possible extent.}'' So, for the current standard, process or
communication failures are to be handled as errors, and the behavior
of the MPI application after an error has been returned is left
unspecified by the standard. However, the standard does not prevent
implementations from going beyond its requirements, and on the contrary,
encourages high-quality implementations to \emph{return} errors once a
failure is detected. Unfortunately, most of the implementations of the 
MPI Standard have taken the path of considering process failures as 
unrecoverable errors, and the processes of the application are most 
often killed by the runtime system when a failure hits any of them, 
leaving no opportunity for the user to mitigate the impact of 
failures. 

%of MPI for MPI-3. One of the workgroups is dedicated to propose a
%standard form of MPI-supported fault tolerance. The proposal outlines
%a method of run-through stabilization which allows the application to
%acknowledge and repair communications, both collectively and between
%specific ranks in a point-to-point way~\cite{Hursey11MPI3FT}. The
%emphasis of the proposal is a set of "validation" functions which the
%application is required to call to repair and re-enable communication within
%an MPI communicator containing a failed process. To repair point to
%point wildcard receives, the application needs to collectively call the function
%MPI\_COMM\_REENABLE\_ANY\_SOURCE. To repair collective communication
%within a communicator, the application needs to call the function
%MPI\_COMM\_VALIDATE.  These functions give the MPI implementation an
%opportunity to acknowledge failures and discover or ensure that other
%MPI processes also acknowledge the same failures. It also gives the
%MPI library a chance to repair communication channels between
%remaining processes, optimizing communication topologies if possible
%and necessary.
%
%While this method of fault tolerance is sufficient for \abft, it is
%not without its drawbacks. The calls necessary to recover from
%collectives incur a non-trivial overhead even during the fault free
%case. MPI\_COMM\_VALIDATE requires a distributed consensus algorithm
%which is currently best implemented at log
%scale~\cite{Hursey11LogConsensus}. While this level of overhead is
%better than the current state of the art of periodic checkpointing, it
%still presents a significant cost that not all applications want or
%need to pay to check the validity of the communicators. Most
%importantly, this proposal does not yet include process recovery,
%which is left to a future proposal to the MPI forum.

% Discuss issues in general with FT-MPI like approaches, besides the 
% sheer problem of standard adoption

% Explain why it is not believed that ABFT can perform without REPLACE 
% or BLANK, or leave it for next section ?

\section{Related Work}\label{sect:related}

% Why coordinated checkpointing is no longer scalable

The current industry standard for failure handling is rollback recovery with
periodic checkpoints to disk, and many libraries have been implemented to
support this behavior~\cite{Duell:tr, Litzkow:1997wd, Plank:1994wz,
Zhong:2001tq}. This has been an effective recovery model for many years and
continues to be an area of research to prolong its
usefulness~\cite{BautistaGomez:2011hg, BuntinasFGCS2008, Elnozahy:2002p3769},
despite the increase in fault frequency and the hierarchization of the
underlying hardware architecture.  However as machines continue to scale,
concerns have been raised about the scalability of rollback
recovery~\cite{Cappello:2009hs}.  The time spent performing recovery operations
is expected to exceed the MTBF in the next generation of supercomputers, causing
any large-scale applications to enter a cycle of recovery where no useful
computation occurs.

% Resilience workshop report

In February 2012, the Department of Defense and Department of Energy conducted
research~\cite{Daly:2012vg} to outline the necessity for resilience at extreme
scale, specifically for exascale computing. They affirmed the likelihood that
exascale systems will have a shorter MTBF than existing systems. They also
recommended that ``a `light-weight' approach, i.e., effective and easy to
implement, is preferable to a `full-featured' alternative.'' By attempting fewer
features, each library can focus on creating a scalable and efficient
implementation while promoting simplicity and reducing unnecessary features. The
agencies also discussed possible redundancy approaches (discussed
in~\cite{BosilcaINRIARep7950, Ferreira:2011fb}), describing them as ``not
practical because they require 2-3x more energy''. The price of redundancy must
not only include the obvious loss of computing time from executing duplicate
codes across multiple processors, but also the up-front costs of purchasing
extra hardware on which to execute the redundancy.

% Describe ABFT

Out of these problems have arisen a new class of algorithms that allow different
methods of resilience. These algorithms support what is called Algorithm Based
Fault Tolerance (ABFT). They have been designed to support a recovery model that
does not require all processes to participate in the recovery simultaneously or,
perhaps, ever at all. Huang and Abraham first explored these
algorithms~\cite{Huang:1984kt} for soft errors in systolic arrays, but research
has expanded to create ABFT techniques for many algorithms~\cite{Du:2012je, Plank:1995hv}.

% RTS Proposal

To support these new algorithms, applications require support from their
libraries. One of the most popular communication libraries used by large scale
applications is the Message Passing Interface (MPI), the contents of which are
determined by the MPI Forum. The MPI Forum has previously considered proposals
to add fault tolerance capabilities to the MPI Standard. In 2011, a proposal was
brought forth~\cite{Hursey2011RTS} titled ``Run-Through Stabilization'' which
discussed a wide range of fault tolerance capabilities to provide for users by
defining the behavior of MPI following a process failure as well as introducing
new constructs to provide strong consensus between processes. The MPI Forum
decided to reject the proposal, but the work led to more research in the field
of resilience in MPI, including the work being proposed in
Section~\ref{sect:ulfm}.

% FT-MPI

Outside of the MPI Forum, work has been done to provide fault tolerance to MPI
applications. FT-MPI~\cite{FaggFTMPI} was an MPI-1 compliant implementation of
the MPI Standard which added new capabilities for fault tolerance. It included
automatic recovery models for failures including: shrinking MPI communicators to
automatically remove failed processes, leaving holes in the communicators while
allowing communication to continue, and destroying and rebuilding the
communicators to retain communication patterns and groups. FT-MPI was never
adopted into the MPI Standard but continued as a research project for many
years. 

