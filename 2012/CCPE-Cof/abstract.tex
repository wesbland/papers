\begin{abstract}

  Most predictions of Exascale machines picture billion way parallelism,
  encompassing not only millions of cores, but also tens of thousands of
  nodes. Even considering extremely optimistic advances in hardware reliability,
  probabilistic amplification entails that failures will be
  unavoidable. Consequently, software fault tolerance is paramount to
  maintain future scientific productivity.  Two major problems hinder ubiquitous
  adoption of fault tolerance techniques: 1) traditional checkpoint based
  approaches incur a steep overhead on failure free operations and 2) the
  dominant programming paradigm for parallel applications (the MPI Standard)
  offers extremely limited support of software-level fault tolerance approaches.
  % The International Exascale Software Project roadmaps predicts, as soon as
  % 2014, billion way parallel machines encompassing not only millions of cores,
  % but also tens of thousands of nodes. Even considering extremely optimistic
  % advances in hardware reliability, probabilistic amplification entails that
  % failures will be unavoidable. Consequently, software fault tolerance is of
  % paramount importance to maintain future scientific productivity. Currently,
  % two major problems hinder ubiquitous adoption of fault tolerance techniques:
  % 1) traditional checkpoint based approaches incur a steep overhead on failure
  % free operations and 2) the dominant programming paradigm for parallel
  % applications (the MPI Standard and its implementations) offers extremely
  % limited support of software-level fault tolerance approaches.
  In this paper, we present an approach that relies exclusively on the features
  of a high quality implementation, as defined by the current MPI Standard, to
  enable advanced forward recovery techniques, without incurring the 
  overhead of customary periodic checkpointing.
  With our approach, when failure strikes, applications regain control to make
  a checkpoint before quitting execution. This checkpoint is in
  reaction to the failure occurrence rather than periodic. This checkpoint 
  is reloaded in a new MPI application, which restores a sane environment 
  for the forward, application-based recovery technique to repair the failure 
  damaged dataset.
  The validity and performance of this approach are evaluated on large scale
  systems, using the QR factorization as an example.
\end{abstract}
