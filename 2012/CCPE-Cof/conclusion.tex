\section{Concluding Remarks}

In this paper, we presented an original scheme to enable forward
recovery using only features of the current MPI Standard.
Rollback recovery, which relies on periodic checkpointing, has a variety
of issues. The ideal period between checkpoints, a critical parameter, is
particularly hard to assess. Too short a period wastes time and
resources on unnecessary Input/Output. Overestimating the period results
in dramatically increasing the lost computation when returning to the
distant last successful checkpoint. Although Checkpoint-on-Failure
involves checkpointing, it takes checkpoint images at optimal times by
design: only after a failure has been detected. This small modification
enables the deployment of \abft techniques, without requiring a complex,
unlikely to be available MPI implementation that itself survives
failures. The MPI library needs only to provide the feature set of a
high quality implementation of the MPI Standard: the MPI communications
may be dysfunctional after a failure, but the library must return
control to the application instead of aborting brutally.

We demonstrated, by providing such an implementation in \ompi, that this
feature set can be easily integrated without noticeable impact on
communication performance. We then converted an existing \abft QR
algorithm to the \cof protocol. Beyond this example, the \cof protocol
is applicable on a large range of applications that already feature an
\abft version, for example the dense direct solvers LLT,
LU~\cite{fthpl2011} and the dense iterative solver
CG~\cite{Chen:2005:FTH:1065944.1065973}. Similarly, \abft algorithms
exist for sparse computation~\cite{Shantharam:2012:FTP:2304576.2304588}.
Beside \abft, many master-slave and iterative methods enjoy an extremely
inexpensive forward recovery strategy where the damaged domains are
simply discarded, and therefore can also benefit from the \cof protocol.

The performance on the Kraken supercomputer reaches
90\% of the non-fault tolerant algorithm, even when including the cost
of recovering from a failure (a figure similar to regular, non-compliant MPI \abft). In addition, on a platform featuring node local storage, the
\cof protocol can leverage low overhead checkpoints (unlike rollback
recovery that requires remote storage). To the extreme, the cost of 
checkpointing can be completely avoided when the application uses a 
master process to actively retain the dataset in memory during the MPI 
restart. 

The MPI standardization body, the MPI Forum, is currently considering
the addition of new MPI constructs, functions and semantics to support
fault-tolerant applications\footnote{https://svn.mpi-forum.org/trac/mpi-forum-web/wiki/FaultToleranceWikiPage}. While these additions may decrease the cost
of recovery, they are likely to increase the failure-free overhead on
fault tolerant application performance. It is therefore paramount to
compare the cost of the \cof protocol with prospective candidates to
standardization on a wide, realistic range of applications, especially
those that feature a low computational intensity.

% unlike dense factorizations where the size of the dataset incurs a cubic
% amount of computation, while checkpoint cost is always linear.
