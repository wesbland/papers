\section{Related Work} \label{sect:related}

Many groups have worked on fault tolerant runtimes in the past, each
proposing different methods. Most of these runtimes focus on
employing checkpoint/restart, but a few also use new ideas to recover
from process failures.

% Charm++ & Adaptive MPI

Charm++ is an object-oriented parallel programming language developed at the
University of Illinois at Urbana-Champaign~\cite{Kale93charm++}. In addition to
message passing, it also performs load balancing, process migration and other
interesting properties due to the fact that it treats each process as an
individual object, called ``chares'', that can be saved and moved at any time.
This has lead to work which leverages these chares to provide fault tolerance
guarantees~\cite{ZhengFTC}, including an MPI implementation which uses Charm++
as its runtime~\cite{ampiJournal}. Our work diverges from Charm++ because they
focus primarily not on MPI, but a separate programming environment of their own
invention.

% FT-MPI

FT-MPI extended the MPI semantics provided by the MPI standard to include fault
tolerance. This enabled application developers to adapt their applications
without the need to rewrite them using an entirely different message passing
system~\cite{fagg2000ft}. FT-MPI could withstand $n-1$ process failures in a job
of size $n$. This is similar to the work being done in this paper as both are
designed to recover from arbitrary fail-stop failures. However, FT-MPI only
provides this semantic to the functions supported by the version 1.2 of the MPI
standard. Our work is built in conjunction with \ompi, which has ongoing
development and includes support for the most recent MPI standard (2.2).

% Other MPI runtimes?

% MPI 3 FT Run-Through Stabilization, Process Recovery

The MPI forum is currently examining options for the future direction of MPI for
MPI-3. One of the workgroups is dedicated to propose a standard form of
MPI-supported fault
tolerance~\footnote{https://svn.mpi-forum.org/trac/mpi-forum-web/wiki/FaultToleranceWikiPage}.
The proposal outlines a method of run-through stabilization which allows the
application to acknowledge and repair communications, both collectively and
between specific ranks in a point-to-point way~\cite{Hursey11MPI3FT}. The
emphasis of the proposal is a set of "validation" functions which the
application is required to call to repair and re-enable communication within an
MPI communicator containing a failed process.
% To repair point to point wildcard receives, the application needs to
% collectively call the function MPI\_COMM\_REENABLE\_ANY\_SOURCE. To repair
% collective communication within a communicator, the application needs to call
% the function MPI\_COMM\_VALIDATE.
These functions give the MPI implementation an opportunity to acknowledge
failures and discover or ensure that other MPI processes also acknowledge the
same failures. It also gives the MPI library a chance to repair communication
channels between remaining processes, optimizing communication topologies if
possible and necessary.

While this method of fault tolerance is sufficient for \abft, it is not without
its drawbacks. The calls necessary to recover from collectives incur a
non-trivial overhead even during the fault free case. MPI\_COMM\_VALIDATE
requires a distributed consensus algorithm which is currently best implemented
at log scale~\cite{Hursey11LogConsensus}. While this level of overhead might
exhibit better performance
than the current state of the art of periodic checkpointing, it still presents a
significant cost that not all applications want or need to pay
% to check the validity of the communicators.
 Also, this proposal does not yet include process
recovery, which is left to a future proposal to the MPI forum.

The work in this paper could be extended to implement the proposal from the
MPI forum if accepted. However, it also includes more flexible options for the user when
selecting a behavior to handle failures.
